\documentclass{article}
\usepackage[persian, group]{hw}

\title{پروپوزال}
\semester{نیم‌سال دوم ۰۰-۰۱}
\course{آز سخت‌افزار-گروه ۳}
\teacher{دکتر اجلالی}

\addauth{علی حاتمی تاجیک}{a.hatam008@gmail.com}{98101385}
\addauth{امیرمحمد عیسی‌زاده}{amirmohammadisazadeh@gmail.com}{98106807}
\addauth{محمدحسین قیصریه}{mgheysariyeh@gmail.com}{97106238}

\begin{document}
\heading
\header
\allowdisplaybreaks
\tableofcontents
\pagebreak

\section{هدف}
هدف این پروژه ساخت کنترلری است که با استفاده از حالت‌ها\footnote{\lr{Gestures}} و حرکت‌های دست بتوان
یک رایانه‌ را کنترل کرد. این عملیات به کمک پردازش تصویر گرفته شده از دست کاربر که مکان و جابجایی های آن به حرکت نشانگر موشواره و عملیات‌های مربوط به آن منجر می‌شود. در کنار این تعدادی حالات قابل تعیین برای کاربر برای روانتر شدن داشته رابط کاربری وجود داشته باشد. در این پروژه نیازی به ماژول شدت صوت نیست و تمام حالت‌ها به وسیله پردازش تصویر قابل دستیابی هستند.

\section{بررسی طراحی}
برای این پروژه دو طراحی\footnote{\lr{Design}} کلی می‌توان متصور بود که در ادامه بررسی خواهند شد. \footnote{البته‌ طراحی‌های هوشمندانه‌تر و فناورانه‌تر نیز قابل تصور است، برای مثال با استفاده از یک فضای خازنی و بدون استفاده از دوربین پردازش‌ها انجام بگیرد تا بسته به تغییر شرایط محیط، نور و ... کارایی بهتری داشته باشد اما از آنجایی که در صورت پروژه کارفرما درخواست ساخت ماژول با استفاده از تکنولوژی پردازش تصویر را داشته است به طراحی‌های منحصر به همین حوزه بسنده شده است.}

\subsection{طراحی با پردازش روی ماشین مقصد}
	در این طراحی محصول نهایی تنها تصویر را از محیط جمع‌آوری کرده و
	برای ماشین مقصد ارسال می‌کند. بار پردازش این داده‌ها بر روی ماشین مقصد خواهد بود که به همین خاطر دستگاه تنها روی ماشین‌هایی که قدرت  پردازشی بالایی دارند (مثل رایانه‌های شخصی و سرورها) قابل استفاده هستند. از طرفی ارسال این اطلاعات به رایانه باعث می‌شود تا محصول نهایی باریک و سبک باشد. همچنین می‌توان نرم‌افزار آنرا به تنهایی به فروش رساند و از سخت‌افزار آن تنها یک بخش انتخابی\footnote{\lr{Optional}} باشد و هر کاربری با استفاده از یک دوربین معمولی و پیکربندی
	\lr{\footnote{Configuration}} 
	مناسب آن بتواند از قابلیت‌های آن استفاده کند.
\\
	مزایا:
	\begin{itemize}
		\item وزن و ابعاد محصول نهایی کوچکتر است.
		\item هزینه تمام شده برای محصول نهایی کمتر است.
		\item امکان فروش به صورت نرم‌افزار صرف را دارد.
		\item می‌تواند به صورت بی‌سیم مورد استفاده قرار بگیرد.
	\end{itemize}
	معایب:
	\begin{itemize}
		\item تنها روی ماشین‌های قدرتمند قابل استفاده است (برای مثال قابل استفاده بر روی تلوزیون‌های هوشمند و کنسول‌های بازی نیست).
		\item مقدار قابل توجهی از منابع کاربر را به صورت مداوم مشغول می‌کند.
	\end{itemize}
	
\subsection{طراحی با پردازش مجزا روی محصول}
در این نوع طراحی باید یک هسته پردازشی قدرتمند با مقدار کافی حافظه در کنار 
دوربین قرار بگیرد تا پردازش به صورت بی‌درنگ\footnote{\lr{Real-Time}} صورت بگیرد و نتایج حاصل از پردازش تصاویر با پروتکل‌های خاص\footnote{\lr{Human Interface Device (HID)}} به ماشین مقصد ارسال شود و در آنجا با استفاده از یک درایور تفسیر شوند. در این طراحی کاربر لزوما باید سخت‌افزار را خریداری کرده که این می‌تواند ارزش‌افزوده زیادی ایجاد کند. \\
	مزایا:
	\begin{itemize}
		\item ماشین مقصد نیاز به قدرت پردازشی (زیادی) ندارد.
		\item می‌توان آنرا به تمام ماشین‌هایی که به موشواره و صفحه‌کلید نیاز دارند متصل کرد.
		\item ارزش‌ افزوده بیشتر برای محصول نهایی متصور است.
		\item انحصار نسبی محصول برای شرکت سازنده حفظ می‌شود.
	\end{itemize}
	معایب:
	\begin{itemize}
		\item به خاطر پردازشی که انجام می‌دهد استفاده از باطری‌ به صرفه نیست و لزوما باید از کابل برای اتصال استفاده شود.
		\item وزن و هزینه محصول نهایی بیشتر خواهد بود.
	\end{itemize}

\section{بررسی جزئیات}
\subsection{بستر پردازشی}
\begin{itemize}
	\item طراحی اول: در طراحی اول نیازی به یک بستر پرقدرت برای پردازش نداریم
	و یک برستر پردازشی به اندازه‌ای که بتواند تصاویر را از سنسور دریافت کرده و آنها را به صورت دیجیتال به کامپیوتر مقصد ارسال کند کفایت می‌کند. برای این کار می‌توان از برد \lr{ESP32} که دارای ارتباطات \lr{Wi-Fi} و بلوتوث است استفاده کرد. انرژی این برد را می‌توان هم به وسیله باطری و هم به وسیله اتصال  مستقیم به رایانه مقصد تامین کرد (به خاطر محدودیت‌های باطری‌ از کابل برای توان بخشی این ماژول استفاده خواهد شد).
	
	\item طراحی دوم: در این طراحی نیاز به قدرت پردازشی زیادی داریم و برد‌های خانواده آردوینو\footnote{\lr{Arduino}} پاسخگوی این قدرت پردازشی نخواهند بود و گزینه مناسب استفاده از خانواده رزبری‌پای\footnote{\lr{RPi-Raspberry Pi}} است.
	
	البته تمام اعضای این خانواده مناسب کار نیستند. همانطور که بالاتر اشاره شد تامین انرژی این طراحی نمی‌تواند به وسیله باطری انجام شود و حتما باید به یک منبع تغذیه خارجی متصل باشد و ارتباطات باید به وسیله سیم انجام بگیرد. مشکلی که در اینجا به وجود خواهد آمد این است که در برخی خانواده‌ها پورت‌های \lr{USB} آن از قابلیت \lr{OTG}\footnote{\lr{On The Go}} پشتیبانی نمی‌کنند (به خاطر هابی که وجود دارد و همینطور استفاده از آن درگاه‌ها به عنوان درگاه اترنت) و تنها اعضای خاصی این قابلیت‌ را دارند که می‌توان از \lr{RPi Zero} نام برد که در کنار قدرت پردازشی مناسبی که دارد اندازه کوچکی نیز دارد و مصرف توان آن هم خیلی زیاد نیست.
	\footnote{به خاطر اینکه بردی که دانشگاه قادر است در اختیار دانشجویان قرار دهد یک از قابلیت \lr{OTG} پشتیبانی نمی‌کند و خرید یک برد رزبریپای جداگانه هزینه‌های ساخت‌ را بالا می‌برد ترجیح بر این است که از طراحی اول استفاده شود.}
\end{itemize}

\subsection{تصویر}
برای دریافت تصویر در طراحی اول از ماژول دوربین که همراه با برد \lr{ESP32} ارائه می‌شود می‌توان بهره برد. می‌توان فیلتر مادون قرمز آنرا نیز جدا کرد تا دوربین دید در شب داشته باشیم   و با یک سورس مادون قرمز کاربری دستگاه را در هنگام نبود نور مرئی نیز حفظ کنیم.

در طراحی دوم باید از ماژول‌های دوربین طراحی شده برای خانواده رزبریپای استفاده کرد. برای این کار می‌توان از حسگر ۵ مگاپیکسلی \lr{OV5647 NoIR} استفاده کرد که به همراه چراغ‌های مادون قرمز هم ارائه می‌شود.

\subsection{پردازش تصویر}
برای پردازش تصویر دریافتی از قسمت قبل از یک برنامه تحت زبان برنامه نویسی پایتون و با استفاده از کتابخوانه‌های
\lr{OpenCV2} و
\lr{MediaPipe}
استفاده می‌شود که این کتابخانه‌ها با استفاده از شبکه‌های عصبی که از قبل روی تعداد زیادی تصاویر تمرین داده شده است حالت‌های دست تشخیص داده می‌شوند. این حالت‌ها در گزارش هفته اول بررسی خواند شد.

\subsection{اتصال به سیستم}
همانطور که در طراحی‌ها هم گفته شد در طراحی اول اتصال به سیستم 
از طریق وایفای انجام میگیرد و تصاویر به رایانه منتقل می‌شوند.

در طراحی دوم این اتصال از طریق یک کابل نوع \lr{C} استفاده می‌شود تا اطلاعات و توان از طریق آن منتقل شود.

\subsection{\lr{Driver}}
با استفاده از کتابخانه \lr{pyautogui} می‌توان تغییرات موشواره و کیبورد را به رایانه منتقل کرد. در طراحی اول چون پردازش روی همان رایانه انجام می‌شود این ماژول مستقیما به پردازش تصویر متصل می‌شود و فرامین را اجرا میکند.

در طراحی دوم اما، از آن جهت که پردازش روی دستگاه انجام می‌شود سیگنال‌های پردازش شده از طریق \lr{USB} به رایانه منتقل می‌شود و با استفاده از پروتکل‌های استاندارد جهانی (مثل \lr{HID}) این سیگنال‌ها مانند سیگنال‌های \lr{trackpad} به رایانه‌ها منتقل خواهند شد.

 
\section{تخمین هزینه}
\begin{itemize}
	\item طراحی اول:
	
\begin{table}[h]
\centering
\begin{tabular}{|c|c|c|}
\hline
ردیف & پارت‌نامبر      & هزینه (ریال) \\ \hline
1    & \lr{ESP32-Camera}    & 2,000,000    \\ \hline
2    & \lr{3D Printed Case} & 2,500,000    \\ \hline
3    & \lr{Micro USB Cable} & 150,000      \\ \hline
\end{tabular}
\end{table}

که با درنظر گرفتن حاشیه امن هزینه تمام شد  5,000,000 ریال خواهد بود.
	
	\item طراحی دوم:
	
	% Please add the following required packages to your document preamble:
% \usepackage{graphicx}
\begin{table}[h]
\centering
\begin{tabular}{|c|c|c|}
\hline
ردیف & پارت‌نامبر                                               & هزینه (ریال) \\ \hline
1    & \lr{Raspberry Pi Zero W}
 & 16,450,000   \\ \hline
2    & \lr{Raspberry Pi Camera Module (NoIR) with 5MP OV5647 sensor} & 5,500,000    \\ \hline
3    & \lr{3D Printed Case                                         } & 3,500,000    \\ \hline
4    & \lr{Micro USB to Type-C Cable                               } & 500,000      \\ \hline
\end{tabular}%
\end{table}
\end{itemize}

که با در نظر گرفتن حاشیه امن قیمت نهایی 28,000,000 ریال خواهد بود.

\section{زمانبندی}
\begin{table}[h]
\centering
\begin{tabular}{|c|c|}
\hline
زمان       & برنامه                            \\ \hline
هفته اول   & طراحی حالات و مقدمات پردازش تصویر \\ \hline
هفته دوم   & پیاده سازی پردازش تصویر           \\ \hline
هفته سوم   & تهیه قطعات و پیکربندی کردن آنها   \\ \hline
هفته چهارم & اتصال به سیستم و تست محصول نهایی  \\ \hline
\end{tabular}
\end{table}

\end{document}
