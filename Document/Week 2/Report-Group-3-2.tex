\documentclass{article}
\usepackage[persian, group]{hw}

\title{پروپوزال}
\semester{نیم‌سال دوم ۰۰-۰۱}
\course{آز سخت‌افزار - گروه ۳}
\teacher{دکتر اجلالی}

\addauth{علی حاتمی تاجیک}{a.hatam008@gmail.com}{98101385}
\addauth{امیرمحمد عیسی‌زاده}{}{}
\addauth{محمدحسین قیصریه}{}{}

\begin{document}
\heading
\header
\allowdisplaybreaks
\tableofcontents
\pagebreak

\section{هدف}
هدف این پروژه ساخت کنترلری است که با استفاده از حالت‌ها\footnote{\lr{Gestures}} و حرکت‌های دست بتوان
یک رایانه‌ را کنترل کرد. 

\section{بررسی طراحی}
برای این پروژه دو طراحی\footnote{\lr{Design}} کلی می‌توان متصور بود که در ادامه بررسی خواهند شد. \footnote{البته‌ طراحی‌های هوشمندانه‌تر و فناورانه‌تر نیز قابل تصور است، برای مثال با استفاده از یک فضای خازنی و بدون استفاده از دوربین پردازش‌ها انجام بگیرد تا بسته به تغییر شرایط محیط، نور و ... کارایی بهتری داشته باشد اما از آنجایی که در صورت پروژه کارفرما درخواست ساخت ماژول با استفاده از تکنولوژی پردازش تصویر را داشته است به طراحی‌های منحصر به همین حوزه بسنده شده است.}

\subsection{طراحی با پردازش روی ماشین مقصد}
	در این طراحی محصول نهایی تنها تصویر (و صوت) را از محیط جمع‌آوری کرده و
	برای ماشین مقصد ارسال می‌کند. بار پردازش این داده‌ها بر روی ماشین مقصد خواهد بود که به همین خاطر دستگاه تنها روی ماشین‌هایی که قدرت  پردازشی بالایی دارند (مثل رایانه‌های شخصی و سرورها) قابل استفاده هستند. از طرفی ارسال این اطلاعات به رایانه باعث می‌شود تا محصول نهایی باریک و سبک باشد. همچنین می‌توان نرم‌افزار آنرا به تنهایی به فروش رساند و از سخت‌افزار آن تنها یک بخش انتخابی\footnote{\lr{Optional}} باشد و هر کاربری با استفاده از یک دوربین معمولی و پیکربندی
	\lr{\footnote{Configuration}} 
	مناسب آن بتواند از قابلیت‌های آن استفاده کند.
\\
	مزایا:
	\begin{itemize}
		\item وزن و ابعاد محصول نهایی کوچکتر است.
		\item هزینه تمام شده برای محصول نهایی کمتر است.
		\item امکان فروش به صورت نرم‌افزار صرف را دارد.
		\item می‌تواند به صورت بی‌سیم مورد استفاده قرار بگیرد.
	\end{itemize}f
	معایب:
	\begin{itemize}
		\item تنها روی ماشین‌های قدرتمند قابل استفاده است (برای مثال قابل استفاده بر روی تلوزیون‌های هوشمند و کنسول‌های بازی نیست).
		\item مقدار قابل توجهی از منابع کاربر را به صورت مداوم مشغول می‌کند.
	\end{itemize}
	
\subsection{طراحی با پردازش مجزا روی محصول}
در این نوع طراحی باید یک هسته پردازشی قدرتمند با مقدار کافی حافظه در کنار 
دوربین قرار بگیرد تا پردازش به صورت بی‌درنگ\footnote{\lr{Real-Time}} صورت بگیرد و نتایج حاصل از پردازش تصاویر (و اصوات) با پروتکل‌های خاص\footnote{\lr{Human Interface Device (HID)}} به ماشین مقصد ارسال شود و در آنجا با استفاده از یک درایور تفسیر شوند. در این طراحی کاربر لزوما باید سخت‌افزار را خریداری کرده که این می‌تواند ارزش‌افزوده زیادی ایجاد کند. \\
	مزایا:
	\begin{itemize}
		\item ماشین مقصد نیاز به قدرت پردازشی (زیادی) ندارد.
		\item می‌توان آنرا به تمام ماشین‌هایی که به موشواره و صفحه‌کلید نیاز دارند متصل کرد.
		\item ارزش‌ افزوده بیشتر برای محصول نهایی متصور است.
		\item انحصار نسبی محصول برای شرکت سازنده حفظ می‌شود.
	\end{itemize}
	معایب:
	\begin{itemize}
		\item به خاطر پردازشی که انجام می‌دهد استفاده از باطری‌ به صرفه نیست و لزوما باید از کابل برای اتصال استفاده شود.
		\item وزن و هزینه محصول نهایی بیشتر خواهد بود.
	\end{itemize}

\section{بررسی جزئیات}
\subsection{بستر پردازشی}

\subsection{تصویر}

\subsection{پردازش تصویر}

\subsection{اتصال به سیستم}

\subsection{\lr{Driver}}

\section{تخمین هزینه}

\section{زمانبندی}

\section{جمعبندی}

\end{document}
