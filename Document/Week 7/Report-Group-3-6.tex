\documentclass{article}
\usepackage[persian, group]{hw}

\title{گزارش هفته: پیکربندی برد رزبریپای، ارتباط سریال و نوشتن درایور}
\semester{نیم‌سال دوم ۰۰-۰۱}
\course{آز سخت‌افزار - گروه ۳}
\teacher{دکتر اجلالی}

\addauth{علی حاتمی تاجیک}{a.hatam008@gmail.com}{98101385}
\addauth{امیرمحمد عیسی‌زاده}{amirmohammadisazadeh@gmail.com}{98106807}
\addauth{محمدحسین قیصریه}{mgheysariyeh@gmail.com}{97106238}

\begin{document}
\heading
\header
\allowdisplaybreaks
\tableofcontents
\pagebreak

\section{مقدمه}
در هفته‌ای که گذشت، کتابخانه‌های مورد نیاز روی برد \lr{Raspberry Pi} نصب شدند، ماژول ارتباط سریال پیکربندی و متصل شد 
و سپس با نوشتن یک درایور برای سیستم‌عامل‌های مختلف و تست کار به کلیات پروژه به پایان رسید. در این میان چالش‌های زیادی
وجود داشت که در هر بخش به آنها پرداخته شده است.

\section{پیکربندی برد رزبری‌پای}
\subsection{چالش‌ها}
در این بخش نیاز بود تا کتابخانه‌های مورد نیاز برای پردازش تصویر و غیره بارگیری و نصب شوند. کتابخانه \lr{OpenCV} به راحتی
قابل نصب بود. اما چالش اصلی در نصب کردن کتابخانه \lr{Meidapipe} بود که قابلیت نصب مستقیم روی این برد را نداشت.

دستورالعملی که گوگل (طراح کتابخانه) داده بود برای سیستم‌عامل‌های عادی کاربرد داشت که با استفاده از ابزار \lr{bazel} باید کد سورس آن
کامپایل و مورد استفاده قرار می‌گرفت اما این ابزار به سختی روی این برد به خاطر حافظه کم قابل نصب بود. پس از امتحان کردن روش‌های مختلف
و بدون نتیجه بودن همه آنها مجبور به این شدیم تا سیستم‌عامل قبلی که روی حافظه برد بود (یک سیستم‌عامل ۳۲ بیتی) را با سیستم‌عامل جدید ۶۴بیتی
که به تازگی معرفی شده بود تعویض کنیم. با انجام این کار  (تعویض سیستم‌عامل) کتابخانه \lr{Meidapipe} نیز به راحتی روی برد قابل نصب شد.

\begin{figure}
    \centering
    \includegraphics[scale=0.5]{graphics/64bit.jpg}
    \caption{سیستم‌عامل که به نسخه ۶۴ بیتی تغییر کرده است.}
    \label{64bit}
\end{figure}

\subsection{نتایج}
پس از کارهای انجام شده کتابخانه‌ها و برنامه‌های زیر روی برد نصب شدند:
قابل مشاهده است.

\begin{latin}
    \begin{itemize}
        \item NumPy
        \item OpenCV
        \item OpenCV-ControlLib
        \item ffmpeg
        \item MediaPipe
        \item MediaPipe-Hands-Solution
        \item PySerial
    \end{itemize}
\end{latin}
 

\section{ارتباط سریال}
\subsection{چالش‌ها}
در ابتدا که \lr{Pinout} مشخص شد و ماژول \lr{TTL} را متصل کردیم  پیام‌های نامانوسی در سمت دیگر سریال
دریافت می‌کردیم که پس از جست‌و‌جوهای فراوان به این مهم دست‌یافتیم که تغذیه از طریق درگاه \lr{USB}
کافی نبوده و منبع جریان قابل اطمینان‌تری مورد نیاز است.

\subsection{نتایج}
پس از اتصال ایمن سریال و تغییر کدی که در فازهای قبلی تست شده بود به کدی به جای چاپ کردن حالت روی رزبری پای حالت‌های مختلف را از طریق رابط سریال به رایانه مقصد ارسال می‌کند تغییر یافت.


\section{نوشتن درایور ویندوز}
\subsection{چالش‌ها}
یکی از چالش‌ّایی که وجود داشت عدم دسترسی به \lr{On-Screen Keyboard} بود که با دسترسی ادمین به سیستم
این مشکل مرتفع شد. مشکل دیگر این بود که نباید حالت \lr{Busy-Waiting}
اتفاق بیافتد زمانی که رزبری چیزی ارسال نمی‌کند به همین دلیل از یک \lr{sleep}
در زمانی که سریال متصل نیست و یا قطع است استفاده شده است که
تا زمانی که سریال به درستی متصل نشده است مشکلی پیش نیاید.

\subsection{نتایج}
در وضعیت فعلی فعالیت با کیبورد مخصوص به ویندوز کاملا مهیاست و به راحتی می‌توان از آن استفاده کرد. همچنین ماژولها به درستی کار می‌کنند (با اغماض از وضعیت ماژول سریال که به صورت موردی گاها با افت ولتاژ مواجه می‌شود).


\section{جمع‌بندی}
در وضعیت فعلی نرم‌افزار پردازشی (هرچند با سرعت پایین‌تر نسب به نسخه‌ای که روی دسکتاپ تست شده بود) در حال کار است و نتیاج آن از طریق درایوری که نوشته شده است قابل دریافت است. (شکل 
\ref{pic})

\begin{figure}
	\centering
	\includegraphics[scale=0.5]{./graphics/cam-config.jpg}
	\caption{پیکربندی دوربین}
	\label{pic}
\end{figure}

\end{document}
