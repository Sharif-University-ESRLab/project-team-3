\documentclass{article}
\usepackage[persian, group]{hw}

\title{گزارش هفته: تکمیل پردازش تصویر، افزودن چند حالت دست اضافه و راه‌اندازی برد رزبری‌پای}
\semester{نیم‌سال دوم ۰۰-۰۱}
\course{آز سخت‌افزار - گروه ۳}
\teacher{دکتر اجلالی}

\addauth{علی حاتمی تاجیک}{a.hatam008@gmail.com}{98101385}
\addauth{امیرمحمد عیسی‌زاده}{amirmohammadisazadeh@gmail.com}{98106807}
\addauth{محمدحسین قیصریه}{mgheysariyeh@gmail.com}{97106238}

\begin{document}
\heading
\header
\allowdisplaybreaks
\tableofcontents
\pagebreak

\section{مقدمه}
در دو (سه) هفته‌ی گذشته منطق بینایی ماشین بهبود یافتند، چند دستور دیگر به مجموعه دستورات اضافه شد
و همینطور برد رزبری‌پای که در اختیارمان قرار گرفته بود راه‌اندازی و پیکربندی شد. همینطور به خاطر تاخیر در ارسال فروشنده برای دو محصول دوربین و کابل \lr{TTL} که مورد نیاز برای این فاز پروژه بود بالاجبار این بخش از این فاز به هفته بعد موکول شد.

\section{بهبود پردازش تصویر}
در هفته‌هایی که گذشت ماژول
\verb~handdetector.py~
از لحاظ الگورتیمی بهبود‌هایی جزئی برای شناسایی بهتر انگشتان و باز و بسته‌بودن آن داشته است که در نرم بودن تجربه کاربری \footnote{\lr{UX-User Experience}}
تاثیراتی داشت.

همینطور چندین دستور کاربردی نیز به مجموعه دستورات افزوده شده است که به شرح زیر هستند:

\subsection{\lr{Scroll}}
با استفاده از سه انگشت و بالا و پایین بردن آن قابلیت اسکرول کردن روی صفحه افزوده شده است.


\subsection{\lr{Drang \& Drop}}
با استفاده از انگشت اشاره و انگشت میانی زمانی که دو انشگت به هم میرسد و حالت \lr{Pinch} رخ میدهد کرسر هر کجا که قرار دارد در حالت درگ میرود و با رها کردن این دو انگشت دراپ رخ می‌دهد.


\subsection{\lr{Right Click}}
همانند حالتی که کلیک رخ میداد اگر انگشت شست نیز باز باشد به حالت راست کلیک رفته و با به هم زدن انگشت میانی و اشاره راست کلیک رخ میدهد.


\subsection{بهبود \lr{Click}}
کلیک کردن تنها به صورت دبل کلیک قابل انجام بود که با کم کردن فاصله زمانی تشخیص به کلیک تبدیل شده که هم کلیک و هم دابل کلیک را ممکن میکند.

\subsection{\lr{Screenshot}}
با باز کردن انشگت شست و انشگت کوچک یک اسکرین شات از صفحه گرفته شده و روی کلیپ‌برد می‌نشیند.

\section{راه‌اندازی برد رزبری‌پای}
در این هفته علاوه بر کار‌های بالا برد ریزبری‌پای به صورت \lr{Head-less} با استفاده از یک نقطه اتصال وای‌فای و اتصال ssh پیکربندی و راه‌اندازی شده است. با توجه به اینکه کابل تی‌تی‌ال و دوربین از فروشنده به دستمان نرسید ادامه پیکربندی ها به هفته بعد موکول شده است.


\end{document}
