\documentclass{article}
\usepackage[persian, group]{hw}

\title{گزارش هفته: طراحی مقدمات پردازش تصویر}
\semester{نیم‌سال دوم ۰۰-۰۱}
\course{آز سخت‌افزار - گروه ۳}
\teacher{دکتر اجلالی}

\addauth{علی حاتمی تاجیک}{a.hatam008@gmail.com}{98101385}
\addauth{امیرمحمد عیسی‌زاده}{amirmohammadisazadeh@gmail.com}{98106807}
\addauth{محمدحسین قیصریه}{mgheysariyeh@gmail.com}{97106238}

\begin{document}
\heading
\header
\allowdisplaybreaks
\tableofcontents
\pagebreak

\section{مقدمه}
در این هفته کلیات و چیدمان مورد نظر به عنوان بستر ابتدایی برای پردازش تصویر دست کاربر مورد بررسی قرار گرفت. البته برای رعایت اختصار ریز جزئیاتی که در جلسه‌ها بررسی شدند (مانند تمام حالت‌هایی که در نظر گرفته شد و یا اشکالات آن حالت‌ها و چرایی رد آنها) در این گزارش نیامده‌اند و تنها خلاصه‌ای از بخش‌های اصلی که روی آنها توافق شد توضیح داده شده اند.

\section{شناسایی}
چیدمان اولیه دوربین اینطور در نظر گرفته شد که دوربین رو به بالا (آسمان/سقف) و رو به روی مانیتور فرد قرار میگیرد. اگر اینطور باشد با حرکت دست فرد روی در فضای قابل قبول بالای دوربین این حس به کاربر دست خواهد داد که دارد انگشتش را در فضای مانیتور خود تکان می‌دهد.
در این حالت نزدیک بودن یا دوربودن دست فرد بالا و پایین مانیتور را مشخص می‌کند، چپ و راست بودن دست چپ و راست بودن در محیط مانیتور و بالا و پایین تصویر دریافتی از ماژول عمق دست کاربر به سمت مانیتور را نمایش می‌دهد.
انگشت سبابه فرد برای دستوراتی که نیاز به دانستن مووقعیت موشواره در صفحه است به عنوان شاخص در نظر گرفته می‌شود.
تشخیص محل انگشت فرد به این صورت است که بعد از کالیبراسیون ابتدایی که کاربر با توجه به ابعاد دست و محدوده فعال دوربین انجام می‌دهد دست کاربر به یک محیط مستطیلی مطابق به سایز مانیتور وی نگاشت می‌شود تا تجربه‌ای روان (مانند اینکه صفحه لمسی است) به  کاربر القا شود.


در این هفته دلیل نامناسب بودن دیگر چیدمان‌ها نیز بررسی شدند که به خاطر اختصار گزارش نهایی از آوردن آن توضیحات اضافه خودداری شد. 

\section{دستورات}
\subsection{چپ‌کلیک}
 هر کلیک شامل دو بخش کلی است. یکی فشرده شدن آن کلید و دیگری رها شدن آن که برای کلیک کردن‌ها این شاخصه‌ها در نظر گرفته خواهند شد.
 
 در طراحی صورت گرفته برای چپ کلیک زیاد شدن عمق دست کاربر به صورت ناگهانی به عنوان فشردن چپ کلیک تلقی می‌شود و بازگرداندن دست به اندازه یک ترشولد خاص (قابل تعیین توسط کاربر) به عنوان رها کردن کلید موس تفسیر خواهد شد. خوبی این طراحی در این است که با استفاده از آن می‌توان قابلیت درگ اند دراپ را به صورت خیلی واقع انگارانه (زمانی که کاربر عمق دست را در صفحه زیاد میکند مانند این است که فایلی را در صفحه گرفته است و دارد جابجا می‌کند) را به کاربر القا خواهد کرد.
 
 

\subsection{راست کلیک}
برای فشار دادن راست کلیک باز شدن شست ملاک است. به این صورت که انگشت سبابه محل نشانگر موس را نشان می‌دهد و با باز شدن شست کلید راست موس فشرده خواهد شد. با بستن انشگت شست کلید راست موس رها خواهد شد.

\subsection{اسکرول}
اسکرول به این صورت انجام خواهد شد که محل موشواره همان انگشت سبابه است و زمانی که با صورت دو انگشتی دست کابر به بالا حرکت کند اسکرول پایین و اگر به سمت پایین حرکت کند اسکرول بالا انجام خواهد شد (القای حس لمسی بودن).

\subsection{دیگر قابلیت‌ها}
در کنار این قابلیت‌های ابتدایی  که به صورت ثابت ارائه می‌شوند تعدادی حالت دست (مانند چهار حرکت سه انگشتی در فضا و یا حالت‌های ثابت دست مانند مشت کردن دست یا نشان دادن لایک) قابلیت‌های کاستومایزابل جنرال و یا حتی خاص در هر برنامه قابل تنظیم توسط کاربر است. برای مثال می‌توان پینچ کردن کاربر را بستن پنجره فعلی در نظر گرفت و یا مشت کردن برابر با قفل کردن صفحه باشد. یا مثلا در برنامه پاورپوینت زمانی که دست باز کاربر از سمت راست به چپ حرکت کرد یک اسلاید جابجا شویم.

به خاطر برنامه نویسی جنریکی که انجام می‌شود می‌تواند تمام جسچرهای اضافه و برنامه‌های \lr{Third-Party} اضافه را به عنوان یک افزونه برای برنامه \lr{Core} اصلی در نظر گرفت و یا به عنوان آپدیت برای نرم افزاری که در حال اجراست ارائه کرد.
\end{document}
