\documentclass{article}
\usepackage[persian, group]{hw}

\title{گزارش هفته: مقدمات پیاده‌سازی پردازش تصویر}
\semester{نیم‌سال دوم ۰۰-۰۱}
\course{آز سخت‌افزار - گروه ۳}
\teacher{دکتر اجلالی}

\addauth{علی حاتمی تاجیک}{a.hatam008@gmail.com}{98101385}
\addauth{امیرمحمد عیسی‌زاده}{amirmohammadisazadeh@gmail.com}{98106807}
\addauth{محمدحسین قیصریه}{mgheysariyeh@gmail.com}{97106238}

\begin{document}
\heading
\header
\allowdisplaybreaks
\tableofcontents
\pagebreak

\section{مقدمه}
در دو هفته گذشته مقدمات پردازش تصویر مربوط به پروژه شامل تکان دادن موشواره و کلیک کردن پیاده‌سازی شده است که در ادامه گزارشی بر چگونگی انجام کار آمده است.

\section{تشخیص دست}
در فایل \verb!handdetector.py! یک ماژول ابتدایی برای تشخیص دست و شاخصه‌های تشخیصی دست شامل خود دست،‌انگشتان و ۲۰ نشان اختصاصی\footnote{\lr{Landmark}} انگشتان دست برای تشخیص بهتر انگشتان دست
پیاده سازی شده است. این فایل شامل کلاس \verb!HandDetector! است که این وظایف را انجام می‌دهد. نشان‌های اختصاصی تشخیص داده شده از روی تصویر دریافتی در شکل ۱ آمده است.

\begin{figure}
\centering
\includegraphics[scale=0.25]{hand_landmarks.png}
\caption{نشان‌های اختصاصی روی دست}
\end{figure}

\section{تغییر جا و کلیک}
کلیک کردن به این صورت انجام می‌شود که ابتدا باید با بالا آوردن دو انگشت به حالت کلیک کردن وارد شد و سپس با نزدیک کردن انگشت اشاره و انگشت وسط کلیک کرد.

برای تکان دادن موشواره باید تنها انگشت اشاره را بالا آورد و سپس آنرا تکان داد.

\end{document}
